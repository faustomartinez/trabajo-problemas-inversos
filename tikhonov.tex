\documentclass[12pt, oneside]{book}
\usepackage[utf8]{inputenc}
\usepackage{amsmath}
\usepackage{amssymb} 
\usepackage{amsmath}
\usepackage{amsthm}

\title{Métodos iterativos para la solución de problemas inversos discretos}
\author{Rivas Jazmin et. al.}
\date{}

\begin{document}
	\maketitle
	\tableofcontents
	
	
	\chapter{Regularización de Tikhonov}
	La regularización es una técnica utilizada para poder resolver sistemas lineales de la forma $Ax = b$. Como vimos, en el área de procesamiento de imágenes este problema está mal condicionado. Es por esta misma razón que se decide añadir un término regularizante.
	La regularización de Tikhonov es una de las más comúnmente utilizadas. En su presentación más simple, se busca resolver \\[10pt]
	 \begin{equation}
	 	\min_{x} \left( \| \mathbf{A}x - \mathbf{b} \|_2^2 + \lambda \| \mathbf{x} \|_2^2 \right)
	 \end{equation}
	 donde $\lambda \geq 0$ es un parámetro de regularización a determinar.
	 
\end{document}