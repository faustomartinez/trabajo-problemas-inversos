\documentclass[12pt, oneside]{book}
\usepackage[utf8]{inputenc}
\usepackage{amsmath}

\title{Métodos iterativos para la solución de problemas inversos discretos}
\author{Rivas Jazmin et. al.}
\date{}

\begin{document}
	\maketitle
	\tableofcontents
	
	
	\chapter{Introducción}
	\ En este texto se pretende estudiar el problema de enfoque de imágenes. A partir de una imagen borrosa con ruido y una matriz de desenfoque, nuestro objetivo es encontrar nuestra imagen original utilizando su versión borrosa. Este problema se puede formular a través de un sistema lineal $Ax = b$, donde $A$ es la matriz de desenfoque, $b$ es el ruido y $x$ es la imagen a restaurar. Debido a su mal condicionamiento, nos interesan los métodos de regularización para la reconstrucción de las mismas. En particular, nos interesaremos por los métodos de regularización de Krylov. Estos métodos están basados en procesos de proyección sobre un tipo de subespacios muy particulares, de los cuales el método toma el nombre. Este capítulo servirá de introducción a la estructura general de esta clase de métodos.
	
	
	\section{Metodos Iterativos}
	\ A
	\subsection{Metodos estandar}
	\ A
	\subsection{Metodos de Krylov}
	\subsubsection{Metodos de Proyeccion}
	\ J
	\subsection{Algoritmo de Arnoldi}
	\subsubsection{FOM}
	\ K
	\subsubsection{GMRES}
	\ K
	
	
	Fausto implementa algo medio copado c:
	
\end{document}
	
